%Note di Ingegneria del Software
%Sommario: Misurazione Quantitativa, Metriche Software, Valutazione, Qualità di Processo, ISO 900x, Gestione Qualità come funzione aziendale, Strumenti di Valutazione

\cornell{ISO/IEC 14598}{Fornisce il modello di valutazione, tramite \textbf{misurazione quantitativa}}

\cornell{Misurazione Quantitativa}{Il processo tramite cui, secondo regole definite (e metriche), simboli o numeri sono assegnati ad attributi di una entità (N. Fenton)}

\cornell{ISO/IEC 25000:2005}{Ingloba gli standard ISO 9126 e ISO 14598, chiamato anche\\
\textbf{SQuaRE} - Software Product Quality Requirements and Evaluation}

\cornell{Metriche Software}{Qualunque tipo di misura che si relaziona ad un sistema Software: \begin{description}
\item [Programmi]: SLOC (Source Line of Code), cioè uno statement
\item [Impegno]: Giorni/Persona, Ore/Persona
\item [Testo]: Gunning's Fog Index\\
Fog = ((media di parole per frase)+(numero di parole con più di 3 sillabe)) $\cdot$ 0.4
\end{description}\\
Le metriche aiutano a misurare attributi, oltre ad aiutare a fare previsioni ed identificare anomalie.}
\cornell{Esempi di metriche}{ \begin{itemize}
\item Numero di Parametri di procedura (Meno è meglio)
\item Complessità Ciclomatica (Meno è meglio)
\item Dimensione del programma, in numero di linee di codice (Solitamente Meno è meglio, a seconda del linguaggio di programmazione)
\item Numero di messaggi d'errore (Nei Log) (Solitamente di più è meglio, aiutando la manutenibilità ed aumentando il grado di diagnostica interna)
\item Lunghezza del Manuale Utente (Solitamente più lungo e preciso è meglio)
\end{itemize} }

\cornell{Valutazione}{Prodotto $\longrightarrow$ Misurazione (scelta delle metriche) $\longrightarrow$ Valutazione (Interpretazione delle misure) $\longrightarrow$ Accettazione (tramite confronto con i criteri di accettazione) $\longrightarrow$ Giudizio}

\cornell{Qualità Di Processo}{Concentrarsi sulla qualità di prodotto non basta, e non dà risultati ripetibili, conviene concentrarsi sulla qualità del way of working.\\
\begin{itemize}
\item Organizzazione e diffusione interna \textbf{sistematica}
\item Identificazione dei momenti di verifica in itinere
\item Riproducibilità dei risultati
\item Quality Assurance, anche \textbf{proattiva}
\end{itemize}\\
Bisogna inoltre avere \textbf{disposizione al miglioramento} (essere fieri del proprio way of working, ma sapere che è sempre migliorabile.)}

\cornell{ISO 900x}{Certificazione ISO 9001 (Seconda metà anni '90)\\
Valutazione dei fornitori di prodotti o servizi.\\
\textbf{ISO 9000:2005} - Fondamenti e glossario.\\
\textbf{ISO 9001:2000} - Sistema Qualità - Requisiti (Praticamente ISO 9000 calato ai sistemi produttivi)\\
\textbf{ISO 9000-3:1997} - Quality Management and Quality Assurance Standards\\
\textbf{ISO 90003:2004} - Praticamente ISO 9001 applicato ai Software\\
\textbf{ISO 9004:2000} - Guida al miglioramento dei risultati}

\cornell{Gestione Qualità Come Funzione Aziendale}{Funziona trasversalmente a settori e repari e riferisce direttamente alla direzione}

\cornell{Manuale della Qualità}{Documento che definisce il sistema di gestione della qualità di un'organizzazione}

\cornell{Strumenti di Valutazione}{ \begin{description}
\item [SPY] Software Process Assessment And Improvement
\item [CMM-CMMI] Capability Maturity Model Integration
\item [SPICE] Software Process Improvement Capability Determination - Evolutosi poi nello standard ISO/IEC 15504
\end{description}}
