%Note di Ingegneria del Software
%Sommario: Design pattern creazionali, singleton, Builder, AbstractFactory

\cornell{Design Pattern Creazionali}{Danno indicazioni per risolvere problemi di istanziazione}

\cornell{Singleton}{(Diventato un anti-pattern)\\
Assicura \textbf{L'esistenza di un'unica istanza} della classe e dà un punto di accesso globale ad essa.\\
Applicabilità \begin{itemize}
\item Deve esistere una sola istanza di una classe in tutta l'applicazione e deve essere accessibile in modo noto
\item L'istanza deve essere estendibile tramite ereditarietà
\end{itemize}\\
Non voglio fare uso di variabili globali perchè: \begin{itemize}
\item Introducono dipendenze
\item Divengono un "single point of failure"
\end{itemize}\\
I singleton sono classi prive di stato, che contengono solo metodi e nessun attributo dipendente dall'istanza. Se i singleton fossero classi con stato, si incorrerebbe in fenomeni di Aliasing.\\
Si fa riferimento a "Singleto" (S maiuscola), per parlare del pattern, mentre si usa "singleton" (S minuscola) per parlare dell'insieme di caratteristiche del pattern.\\
Noi faremo riferimento alla seconda denominazione, soprattutto all'interno dei diagrammi UML}

\cornell{Singleton: Implementazione}{\begin{itemize}
\item Rendo privato il costruttore, così che l'oggetto non possa essere istanziato da altri
\item Mantengo un riferimento statico all'istanza all'interno della classe
\item Creo un metodo statico nella classe \textbf{che crea l'istanza se il riferimento memorizzato è null}, e restituisce l'istanza statica memorizzata
\end{itemize}
Questa implementazione presenta dei problemi, oltre che dei vantaggi: \begin{itemize}
\item In caso di concorrenza sul metodo statico, potrebbero venirsi a creare due singleton, di cui uno mai usato (dangling reference)
\item Non è semplicemente serializzabile
\item È un'implementazione Lazy, l'istanza del singleton viene creata solo quando necessario; risparmiando risorse.
\end{itemize}\\
Il problema della concorrenza può essere risolto inizializzando il riferimento statico immediatamente, andando però a perdere la laziness.\\
Si possono avere ulteriori migliormanti facendo uso delle classi enumeratore.\\
In questo modo sarà il linguaggio stesso (Java, nella lezione) che ci garantirà l'unicità dell'istanza}

\cornell{Problemi del singleton}{ \begin{itemize}
\item Alla fine di tutto, è solo una variabile globale vista in ottica Object-Oriented
\item Crea problemi con lo unit testing. Nascondendo le dipendenze, io non riesco a fornire al singleton dei mock/stub per effettuare i test.
\end{itemize}\\
Per questi motivi all'oggi il singleton viene considerato un anti-pattern.\\
%TODO: Aggiungere codice da GitHub}

\cornell[Errori Comuni]{UML e Singleton}{Molte volte negli anni passati si sono raffigurati i singleton come classi con un riferimento a se stesse (tramite la freccia).\\
Questa rappresentazione è \textbf{assolutamente sbagliata}}

\cornell{Builder}{Separa la costruzione di un \textbf{oggetto complesso} dalla sua rappresentazione\\
Permette di costruire oggetti complessi senza aver a che fare con costruttori grandi e situazioni di telescoping (tanti costruttori che richiamano un costruttore grande con diverse combinazioni di parametri opzionali o meno)}

\cornell{Builder: Implementazione}{ \begin{itemize}
\item Rendo il costruttore privato o package private
\item Creo una classe "builder" con gli stessi attributi della classe da costruire ed un setter per ogni attributo ed un metodo per costruire l'oggetto finale.\\
Ogni setter dovrà ritornare il riferimento \texttt{this} per permettere la concatenazione di setter.
\end{itemize}\\
%TODO: Esempio di codice da GitHub
In questo modo (almeno in Java), si vengono a creare così delle cosiddette "interfacce fluide"}

\cornell{Vantaggi Builder}{ \begin{itemize}
\item Permette di controllare opzionalità ed obbligatorietà di parametri (tramite ad esempio objects.nonNull())
\item Facilita le modifiche della rappresentazione interna di un prodotto, basta costruire un nuovo builder.
\item Aumenta l'incapsulazione
\item Miglior controllo del processo di costruzione
\begin{itemize}
        \item Tramite la costruzione step-by-step
        \item Accentrando la logica di validazione
\end{itemize}
\end{itemize}\\
L'accentramento della logica di validazione dentro al builder (ed al di fuori della classe costruita) va bene dato che Builder e classe costruita sono fortemente legate}

\cornell{Abstract Factory}{Fornisce un'interfaccia per costruire famiglie di prodotti non collegate tra loro ma le cui componenti di una famiglia vanno usate assieme; il tutto senza andare a creare classi concrete\\
\begin{itemize}
        \item Demanda la costruzione degli oggetti ad una classe terza
\item In questo modo evito errori del tipo "Creo un bottone per mac su una finestra per windows".\\
La factory creerà gli oggetti corretti per me.
\item Ho un unico punto di decisione nel codice, rendendolo configurabile.
\item Devo nascondere i costruttori all'esterno, permettendo la costruzione degli oggetti solo tramite le factory\\
Le factory possono essere delle buone classi singleton.
\end{itemize}}

\cornell{Vantaggi/Svantaggi Factory}{ \begin{itemize}
\item Promuove la consistenza fra prodotti
\item Aumenta la semplicità nell'uso di una famiglia di prodotti
\item Attua l'isolamento dei tipi concreti (in quanto i client vanno solo ad usare interfacce)
\item Diventa però complesso aggiungere nuovi prodotti alle famiglie
\end{itemize}}
