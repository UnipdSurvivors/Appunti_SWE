%Note di Ingegneria del Software
%Sommario: Definizione di Progetto, Attività/Compito, Ingegneria, Prodotto, Best practices

\cornell{Progetto}{Insieme di \emph{attività} e \emph{compiti} con le seguenti proprietà:\begin{itemize}
\item Raggiungano determinati obiettivi con specifiche fissate
\item Hanno date d'inizio e fine ben definite
\item Contano su disponibilità limitate di risorse (tempo, fondi, denaro, strumenti)
\item Nel loro svolgimento consumano risorse.
\end{itemize}\\
(Definizione di H. Kerzner)\\
\textbf{Attenzione:} Attività e compiti non sono la stessa cosa}
\cornell{Attività e compiti}{Le attività sono le cose che vanno fatte (in senso generale), mentre i compiti sono attività assegnate a qualcuno.\\
Tra le attività che devono essere svolte nell sviluppo di un progetto troviamo:\begin{description}
	\item [Pianificazione:] Gestire risorse e responsabilità
	\item [Analisi dei requisiti:] definire \textbf{cosa} si deve fare
	\item [Progettazione:] (design) definire \textbf{come} farlo
	\item [Realizzazione:] \textbf{farlo}:
	\begin{itemize}
		\item Perseguendo \textbf{qualità}
		\item \textbf{Verificando} l'assenza di errori
		\item \textbf{Validando} il risultato rispetto alle attese
	\end{itemize}
\end{itemize}}
\cornell{Ingegneria}{"Applicazione di principi scientifici e matematici a fini pratici" (American Heritage Dictionary)\\
L'ingegneria \emph{applica} principi noti e autorevoli, non ne inventa $\Longrightarrow$ Best practice\\
Il fine pratico è spesso di carattere civile e sociale, comportando responsabilità di carattere etico e professionale.}
\cornell{Ingegneria del software}{Disciplina la realizzazione di prodotti Software abbastanza complessi da richiedere lavoro collaborativo \begin{itemize}
\item Garantendo qualità (\textbf{efficacia})
\item Avendo costi e tempi contenuti (\textbf{efficienza})
\item Lungo l'intero \textbf{ciclo di vita} del software}
\cornell{Prodotto Software}{Si può definire prodotto software uno dei seguenti:\begin{description}
\item [Commessa] Forma, contenuto e funzione sono fissate dal committente
\item [Pacchetto] Forma, contenuto e funzione sono adatte alla replicazione
\item [Componente] Forma, contenuto e funzione sono adatte alla composizione
\item [Servizio] Forma, contenuto e funzione sono fissate dal problema
\end{description}}
\cornell{Ciclo di vita}{Gli stati di avanzamento che il prodotto assume dal concepimento al proprio ritiro}
\cornell{Efficacia}{\textbf{Misura} la capacità di raggiungere gli obiettivi fissati $\Longrightarrow$ grado di conformità}
\cornell{Efficienza}{\textbf{Misura} l'abilità di raggiungere gli obiettivi impiegando meno risorse possibili $\Longrightarrow$ riduzione dello spreco}
\cornell{Best Practice}{Modo di fare noto che abbia dimostrato di garantire i migliori risultati (in termini di efficienza ed efficacia) in circostanze specifiche e note}
\cornell{Utilità di un software}{Un software si dice tanto più utile quanto più è usato}
\cornell{Metrica}{Integrale degli usi (o utenti) nel tempo, misura dell'utilità di un software}
\cornell{Manutenzione}{Nei software con ciclo di vita "lungo", viene effettuata manutenzione (con i propri costi annessi): \begin{description}
\item [Correttiva] Atta alla rimozione di difetti
\item [Adattativa] Raffinamento dei requisiti
\item [Evolutiva] Evoluzione del sistema
\end{description}}
