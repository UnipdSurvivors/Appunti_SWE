%Note di Ingegneria del Software
%Sommario: Flipped classroom sul tema organizzazione, SEMAT

\cornell{Introduzione alla flipped class}{Dobbiamo capire quali milestone è necessario settare, e se settarle partendo dal loro significato, oppure settare semplicemente una data di calendario.\\
È inoltre necessario capire come arrivare a tali mileston, cioè definire l'organizzazione del lavoro.}

\cornell{Milestone}{Punto di calendario a cui dò un significato, un valore di progresso. È un punto di controllo}

\cornell{Chi decide la milestone?}{È un'assunzione di responsabilità libera, che cambia a seconda del gruppo o dell'entità che si prende in carico il progetto}

\cornell{Come pongo le milestone?}{ \begin{itemize}
\item A seconda del loro significato
\item In un punto fisso del calendario
\end{itemize}\\
Devo settare le milestone tenendo conto prima del loro significato, ma senza trascurare la presenza di vincoli temporali imprescindibili}

\cornell{Le milestone nel nostro progetto}{Considerando come riferimento le slides del 16 ottobre, schede SEMAT 1 e come data della Revisione dei Requisiti (deadline) il 16 gennaio 2018, abbiamo 4 fronti su cui lavorare: \begin{itemize}
\item Way of working
\item Requisiti
\item Lavoro (Work nel SEMAT) (Project Management)
\item Software System
\end{itemize}\\
Prendendo come inizio la formazione dei gruppi, nella linea del "way of working" dovremo avere una milestone "principles estabilished" appena dopo la formazione dei gruppi, in cui definiamo a quali obiettivi di norme puntiamo, i principi ispiratori, come cercare le nostre norme.\\
Successivamente vi sarà una fase di "Foundation Estabilished", dove si hanno le fondamenta del lavoro preparate.\\
Prima del 16 gennaio, il way of working dovrà essere "in use", con un tempo sufficiente a poter redarre, secondo tali regole, la documentazione necessaria, ad esempio.\\
Nella linea dei requisiti, i requisiti dovranno essere "acceptable" (accettabili dagli stakeholder) prima del 16 gennaio, con un intervallo di tempo sufficiente a consentire eventuali correzioni di rotta.\\
Il lavoro, (cioè il project management) dovrà essere allo stato "started" molto prima del 16 gennaio, tra la venuta di un "way of working in use" e la definizione dei requisiti come "acceptable".\\ Quindi la gestione del progetto deve essere allo stato "started" abbastanza tempo prima della deadline, in modo che si abbiano dei requisiti accettabili.\\
Il sistema software invece troverà la sua prima fase, architecture selected, \textbf{dopo} il 16 gennaio.}
