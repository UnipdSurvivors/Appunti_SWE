%Note di Ingegneria del Software
%Sommario: Prodotti Documentali, pensieri top-down e bottom-up, studio di fattibilità, analisi e classificazione dei requisiti

\cornell{Prodotti Documentali}{Non posso "raccontare a voce" informazioni collaborative perchè: \begin{itemize}
\item "Verba Volant" - Non può rimanere traccia scritta di ciò che ho detto
\item La discussione orale in stile meeting blocca le persone dal compiere le proprie attività
\end{itemize}\\
Quindi devo avere a disposizione dei formati non invasivi, cioè dei documenti scritti.}
\cornell{Primi prodotti documentali}{\begin{description}
\item [Capitolato D'appalto] Definisce i requisiti del cliente (Il "Cosa fare")
\item [Specifica dei requisiti Software] Il "Come fare" ciò che è definito dal capitolato\\
Comprende: \begin{description}
\item [Studio di fattibilità] Che è un documento riservato al fornitore
\item [Analisi dei requisiti] cioè definisco col cliente ciò che devo fare, con questo documento il fornitore viene formalmente ingaggiato per il lavoro
\end{description}
\item [Ripartizione dei requisiti]
\end{description}}

\cornell{Approccio Funzionale}{È un approccio di tipo top-down.\\
Parte dal sistema completo (il tutto) e lo suddivide in componenti (secondo il principio "divide et impera")\\
È l'approccio più immediato per gli esseri umani.\\
Non aiuta il riuso, quindi è un pensiero da evitare.}
\cornell{Approccio Object-Oriented}{Approccio di tipo bottom-up.\\
Parto da pezzi generici che opportunamente manipolati possono diventare il sistema.}

\cornell{Studio di Fattibilità}{Valutazione dei costi, rischi e benefici. \begin{itemize}
\item Fattibilità tecnico-organizzativa
\item Rapporto costi/benefici
\item Individuazione dei rischi
\item Valutazione delle scadenze temporali
\item Valutazione delle alternative: \begin{itemize}
\item Make Or Buy (Affidarsi al riuso oppure sviluppare ex-novo)
\item Scelte di Architettura
\item Strategie Operative
\end{itemize}
\end{itemize}\\
Questo passaggio ha tempistiche molto strette (Solitamente si ha un mese per Studio di Fattibilità \textbf{e} Analisi dei requisiti)}
\cornell{Tecniche di Analisi: Analisi dei bisogni e delle fonti}{Ci si dedica alla comprensione del dominio\\
Mi metto nei panni dell'utente che ha scritto il capitolato, cercando di esplicitare ciò che è implicito nel dominio. (Possono infatti essere presenti requisiti impliciti)}
\cornell{Tecniche di Analisi: Interazione col Cliente}{Interviste: quando capiamo cose col cliente, queste vanno trascritte nella cosiddetta "minuta col cliente", un documento scritto.\\
La minuta fa da appendice al capitolato ed ha valore contrattuale, quindi va approvato dalle parti.}
\cornell{Tecniche di Analisi: Discussioni Creative e collaborative}{Brainstorming.\\
In ogni brainstorming ci devono essere $n+2$ persone, di cui $n$ discutono, più \begin{itemize}
\item Qualcuno che funga da "scriba", raccogliendo i punti della discussione su cui il gruppo converge
\item Qualcuno che regolamenti la discussione in modo che \begin{itemize}
\item Non ci si parli l'uno sopra l'altro
\item Non si abbia quella persona che "parla sempre"
\item Non si finisca per divagare
\end{itemize}
\end{itemize}}
\cornell{Tecniche di Analisi: Prototipazione}{ \begin{itemize}
\item Interna (Solo per il fornitore)
\item Esterna (Per discussione col cliente)
\end{itemize} }
\cornell{Classificazione Dei Requisiti}{Ordinare i requisiti facilita la comprensione, manutenzione e tracciamento.\\
\begin{itemize}
\item Attributi di prodotto\\
"Cosa devo fare?"\\
Riguardano strettamente il sistema/prodotto
\item Attributi di Processo\\
"Come devo farlo?"\\
Riguarda il "way of working" (che viene solitamente imposto dal committente)
\end{itemize}\\
Altra possibile classificazione (per utilità strategica)
\begin{description}
\item [Obbligatori] Irrinunciabili, per qualche stakeholder
\item [Desiderabili] hanno valore aggiunto riconoscibile, ma non sono strettamente necesari
\item [Opzionali] relativamente utili, oppure contrattabili più avanti
\end{description}\\
Questi requisiti non devono essere in conflitto tra loro e devono essere \textbf{tracciabili} (avere requisiti atomici li rende facilmente verificabili, e quindi tracciabili)}
