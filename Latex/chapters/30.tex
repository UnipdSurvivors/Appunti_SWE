%Note di Ingegneria del Software
%Sommario: Riassunto sugli standard

\pagebreak
\backgroundsetup{contents={}}
\section{Ripasso sugli standard}
Durante questo corso sono stati proposti molti standard che riporterò qui sotto per praticità di consultazione.\\
\subsection{Standard Fondamentali}
Gli standard qui sotto non devono assolutamente essere dimenticati per poter passare l'esame: \begin{description}
\item [ISO/IEC 12207] \textit{Processi di ciclo di vita} - Fornisce una visione ad alto livello dei processi coinvolti nel ciclo di vita di un Software (Primari, Organizzativi, di Supporto);
\item [ISO 9000] \textit{Modello di qualità di processo} - Neutrale rispetto al dominio applicativo;
\item [ISO 9001] \textit{Sistema di gestione qualità} - Praticamente ISO 9000 calato ai sistemi produttivi. È anche una certificazione;
\item [ISO/IEC 9126] \textit{Modello di qualità di un prodotto software};
\item [ISO/IEC 14598] \textit{Linee guida per misurazioni e metriche di un prodotto software};
\item [ISO/IEC 25000] \textit{SQuaRE: \textit{S}oftware product \textit{Qua}lity \textit{R}equirements and \textit{E}valuation} - Ingloba in sè gli standard 9126 e 14598;
\item [ISO/IEC 15504] \textit{SPICE: \textbf{S}oftware \textbf{P}rocess \textbf{I}mprovement \textbf{C}apability d\textbf{E}terminaiton} - Misurazione e valutazione di capability dei processi;
\item [ISO/IEC 15939] \textit{Metriche di processo} - Fornisce un modello di misurazione ed un modello di processo di misurazione
\end{description}

\subsection{Altri Standard}
Questi standard sono collaterali ma collegati a quelli più importanti: \begin{description}
\item [ISO/IEC 90003] \textit{Linee guida per l'applicazione di ISO 9000 al software}
\item [ISO/IEC 9004] \textit{Guida al miglioramento dei risultati}
\end{description}
