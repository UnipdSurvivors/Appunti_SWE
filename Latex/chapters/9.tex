%Note di Ingegneria del Software
%Sommario: Analisi dei rischi, Pianificazione, Baseline/Milestone, Analisi dei requisiti

\cornell{Analisi Dei Rischi}{Devo trovare cosa può "rompere" la mia pianificazione. Vi sono 3 passaggi preliminari \begin{enumerate}
\item Identificare i rischi nei quali potrei incorrere\\
(Cosa potrebbe andare male, realisticamente. Include fattori personali, tecnologici, \ldots)
\item Per ogni rischio, analizzo quanto probabile e imminante questo rischio è, e quanto danno può provocare e quali conseguenze ci sono
\item Organizzo le attività allo scopo di minimizzare l'impatto e la probabilità dei rischi.
\end{enumerate}
In più vi è un passaggio iterativo: \begin{enumerate}
\setcounter{enumi}{3}
\item Controllo Continuativo\\ Guardo ed aggiorno i rischi, raffinando la mia strategia
\end{enumerate}}
\cornell{Pianificazione}{Per ridurre i rischi, si pianifica su segmenti di tempo piccoli e controllabili, così da ridurre conseguenze, impatto e probabilità di questi.}
\cornell{Milestone}{Punto \textbf{nel tempo} con \underline{significato strategico} $ \Longrightarrow $ punto di riferimento}
\cornell{Baseline}{Punto di avanzamento \textbf{del prodotto} $ \Longrightarrow $ È misurabile}
\cornell{Baseline+Milestone}{La milestone va associata a risultati tangibili $ \Longrightarrow $ ogni milestone è associata ad una o più baseline}
\cornell{Analisi dei Requisiti}{Fase importantissima del progetto, oltre che prima attività del progetto software.}
\cornell{Requisito}{\begin{itemize}
\item Un Bisogno (dal punto di vista del richiedente)
\item Capacità di un sistema di adempiere ad una richiesta (visto dal punto di vista dell'offerente)
\end{itemize}}
\cornell{Verifica}{Accertare che l'esecuzione di attività non introduca errori.\\
È un processo di supporto.\\
Si può ricordare con la frase "Did I build the system right?" (Ho fatto le cose come dovuto/nel modo giusto?)}
\cornell{Validazione}{Accertare che il prodotto corrisponda alle attese.\\
Si può ricordare con la frase "Did I build the right system?" (Ho costruito il sistema giusto?)}
\cornell{Verifica E Baseline}{Per sapere dove sono arrivato in una baseline faccio \textbf{molte verifiche} così da avere una misura di avanzamento}
\cornell{Qualifica}{Composta dalle verifiche più la validazione finale (le validazioni sono molto costose, quindi non se ne fanno molte. Solitamente viene fatta solo la finale.)}
\cornell{Attività Necessaria in questa fase}{\begin{itemize}
\item Analisi
\item Piano di Qualifica
\end{itemize}}
\cornell{Inizio Dell'Analisi}{\begin{itemize}
\item Studio dei bisogni e delle fonti\\
Mi metto nell'ottica del committente per capire la natura e l'origine dei bisogni\\
Identifico, specifico (univocamente) e classifico (per importanza, negoziabilità, \ldots) i requisiti.
\item Modellazione concettuale del sistema\\
Penso a come qualcuno da fuori agisce col mio sistema $ \Longrightarrow $ Diagramma dei casi d'uso\\
Vedo \textbf{cosa} fa il sistema per risolvere il problema, non \textbf{come} lo fa.
\end{itemize}}
\cornell{Processi di supporto Implicati}{\begin{itemize}
\item Documentazione
\item Gestione e manutenzione dei prodotti
\begin{itemize}
\item È necessario gestire i requisiti (A scopo di fare controllo di conformità)
\item Gestione della configurazione
\item Gestione dei cambiamenti (È un processo, chiamato "Change management")
\end{itemize}
\end{itemize}}
