%Note di Ingegneria del Software
%Sommario: Versionamento, Pianificazione

\cornell{Cos'è}{Il Controllo Versione (o versioning) è un insieme di tecniche e strumenti atti alla gestione dei cambiamenti nei files.\\
La necessità di poter gestire questi cambiamenti si è fatta sentire molto più nell'era dell'informatica, dove è abbastanza semplice introdurre modifiche ad un software che debbano poi essere rimosse, solitamente a seguito di malfunzionamenti provocati da tali modifiche.}

\cornell{Come Funziona il Versioning: Teoria}{Sfruttando la teoria dei grafi possiamo vedere l'insieme di revisioni di un software come un grafo direzionato ed aciclico, in cui possiamo vedere le seguenti componenti: \begin{description}
\item [Trunk] "Il tronco" Che rappresenta la linea di sviluppo principale, solitamente includendo nuove caratteristiche
\item [Branches] "I rami" che si diramano dal trunk in diversi momenti della sua vita
\end{description}}

\cornell{Branches}{I rami solitamente sono usati per \begin{itemize}
\item Lavorare su una nuova feature senza intaccare la base su cui si basa il branch
\item Lavorare parallelamente al trunk, un classico esempio è il branch di manutenzione di una certa versione del software
\item Lavorare parallelamente ad altri branch, rendendo più rapido lo sviluppo del software
\end{itemize}}

\cornell{Merging}{Quando un lavoro su un branch si considera concluso, vi sono due possibilità: \begin{itemize}
\item Abbandono del branch (in tal caso si parla di "discontinued branch")
\item Merge nel trunk
\end{itemize}
In caso di merge, il contenuto del branch viene unito a quello del trunk, con eventuale gestione di conflitti.}

\cornell{Commits}{Essendo uno strumento anche collaborativo, i software di version control devono garantire l'atomicità delle operazioni, che in termini tecnici vengono dette "commit", in modo da evitare interferenze sui file che potrebbero rovinare il lavoro.}

\cornell{Software di Controllo Versione}{Esistono vari software per fare controllo versione, i più famosi sono git, subversion (svn) e mercurial (hg).}

\cornell{Git}{Nato nel 2005 ad opera di Linus Torvalds, git è un software di controllo versione nato dopo che molti sviluppatori del kernel linux hanno dovuto abbandonare l'accesso al codice sorgente tramite BitKeeper, dato che a detta del proprietario era stato fatto un reverse engineering dei protocolli Bitkeeper (cosa non ammessa)\\
Git è un sistema di versionamento con forte supporto allo sviluppo non lineare, permettendo branching e merging rapidi.\\
Inoltre è dedicato fortemente allo sviluppo distribuito e permette di pubblicare i repository via HTTP, FTP, ssh o rsync in maniera semplice.\\
Git è inoltre estremamente scalabile, è dotato di autenticazione crittografica della cronologia (così che se le vecchie versioni sono cambiate, questo possa essere notato), permette la concatenazione delle proprie componenti ed è dotao di strategie di fusione intercambiabili.}

\cornell{Subversion}{SVN è un altro sistema di controllo versione, creato da CollabNet inc. come successore di CVS (Concurrent Versions System).\\
In questo programma le commit sono vere transazioni atomiche, infatti interrompere una commit lascia il repository in uno stato di incoerenza.\\
Alcune operazioni (come il branching) non dipendono dalla dimensione dei dati\\
È basato su un protocollo client/server che invia solo le differenze in entrambe le direzioni, rendendo i costi di comunicazione proporzionali alla dimensione di tali modifiche.}

\cornell{Conclusione Sul Versionamento}{È assolutamente necessario porsi delle regole su chi deve intervenire su quale componente del progetto/documentazione, in modo da evitare perdite di tempo nella gestione dei conflitti.\\
La storia di un progetto è vitale, e deve essere conservata in maniera quanto più immutabile, se si devono operare modifiche in parti storiche del progetto, dovrebbe essere sempre possibile rintracciare la versione precedente a tale modifica.}

\cornell{Bug, Ticket Trackers e pianificazione}{Un altro utile strumento di supporto per la coordinazione nei progetti è il tracker. Il cui scopo è quello, appunto di "tener traccia" di qualcosa.\\
Solitamente nell'ambito dello sviluppo software si usano bug trackers, per tener conto dei dettagli di bachi rilevati in un software ed i vari procedimenti, commenti e soluzioni proposte riguardo a tali bachi.\\
Un altro esempio, di carattere più generico sono i ticket trackers, dove si tiene conto non solo di bachi software ma anche di altre attività legate al progetto.\\
È comunque vitale avere una persona con l'esperienza necessaria, e la volontà di prendersi responsabilità delle decisioni prese sul progetto. Questo ruolo è solitamente coperto dal project manager.}
